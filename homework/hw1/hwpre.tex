% CS 580 style
% Typical usage (all UPPERCASE items are optional):
%       \input 580pre
%       \begin{document}
%       \MYTITLE{Title of document, e.g., Lab 1\\Due ...}
%       \MYHEADERS{short title}{other running head, e.g., due date}
%       \PURPOSE{Description of purpose}
%       \SUMMARY{Very short overview of assignment}
%       \DETAILS{Detailed description}
%         \SUBHEAD{if needed} ...
%         \SUBHEAD{if needed} ...
%          ...
%       \HANDIN{What to hand in and how}
%       \begin{checklist}
%       \item ...
%       \end{checklist}
% There is no need to include a "\documentstyle."
% However, there should be an "\end{document}."
%
%===========================================================
\documentclass[11pt,twoside,titlepage]{article}
%%NEED TO ADD epsf!!
\usepackage{threeparttop}
\usepackage{graphicx}
\usepackage{latexsym}
\usepackage{color}
\usepackage{listings}
\usepackage{fancyvrb}
%\usepackage{pgf,pgfarrows,pgfnodes,pgfautomata,pgfheaps,pgfshade}
\usepackage{tikz}
\usepackage[normalem]{ulem}
\tikzset{
    %Define standard arrow tip
%    >=stealth',
    %Define style for boxes
    oval/.style={
           rectangle,
           rounded corners,
           draw=black, very thick,
           text width=6.5em,
           minimum height=2em,
           text centered},
    % Define arrow style
    arr/.style={
           ->,
           thick,
           shorten <=2pt,
           shorten >=2pt,}
}
\usepackage[noend]{algorithmic}
\usepackage[noend]{algorithm}
\newcommand{\bfor}{{\bf for\ }}
\newcommand{\bthen}{{\bf then\ }}
\newcommand{\bwhile}{{\bf while\ }}
\newcommand{\btrue}{{\bf true\ }}
\newcommand{\bfalse}{{\bf false\ }}
\newcommand{\bto}{{\bf to\ }}
\newcommand{\bdo}{{\bf do\ }}
\newcommand{\bif}{{\bf if\ }}
\newcommand{\belse}{{\bf else\ }}
\newcommand{\band}{{\bf and\ }}
\newcommand{\breturn}{{\bf return\ }}
\newcommand{\mod}{{\rm mod}}
\renewcommand{\algorithmiccomment}[1]{$\rhd$ #1}
\newenvironment{checklist}{\par\noindent\hspace{-.25in}{\bf Checklist:}\renewcommand{\labelitemi}{$\Box$}%
\begin{itemize}}{\end{itemize}}
\pagestyle{threepartheadings}
\usepackage{url}
\usepackage{wrapfig}
% removing the standard hyperref to avoid the horrible boxes
%\usepackage{hyperref}
\usepackage[hidelinks]{hyperref}
% added in the dtklogos for the bibtex formatting
\usepackage{dtklogos}
%=========================
% One-inch margins everywhere
%=========================
\setlength{\topmargin}{0in}
\setlength{\textheight}{8.5in}
\setlength{\oddsidemargin}{0in}
\setlength{\evensidemargin}{0in}
\setlength{\textwidth}{6.5in}
%===============================
%===============================
% Macro for document title:
%===============================
\newcommand{\MYTITLE}[1]%
   {\begin{center}
     \begin{center}
     \bf
     FS 101\\Software Everywhere\\
     Fall 2013
     \medskip
     \end{center}
     \bf
     #1
     \end{center}
}
%================================
% Macro for headings:
%================================
\newcommand{\MYHEADERS}[2]%
   {\lhead{#1}
    \rhead{#2}
    %\immediate\write16{}
    %\immediate\write16{DATE OF HANDOUT?}
    %\read16 to \dateofhandout
    \def \dateofhandout {September 6, 2013}
    \lfoot{\sc Handed out on \dateofhandout}
    %\immediate\write16{}
    %\immediate\write16{HANDOUT NUMBER?}
    %\read16 to\handoutnum
    \def \handoutnum {2}
    \rfoot{Handout \handoutnum}
   }

%================================
% Macro for bold italic:
%================================
\newcommand{\bit}[1]{{\textit{\textbf{#1}}}}

%=========================
% Non-zero paragraph skips.
%=========================
\setlength{\parskip}{1ex}

%=========================
% Create various environments:
%=========================
\newcommand{\PURPOSE}{\par\noindent\hspace{-.25in}{\bf Purpose:\ }}
\newcommand{\SUMMARY}{\par\noindent\hspace{-.25in}{\bf Summary:\ }}
\newcommand{\DETAILS}{\par\noindent\hspace{-.25in}{\bf Details:\ }}
\newcommand{\HANDIN}{\par\noindent\hspace{-.25in}{\bf Hand in:\ }}
\newcommand{\SUBHEAD}[1]{\bigskip\par\noindent\hspace{-.1in}{\sc #1}\\}
%\newenvironment{CHECKLIST}{\begin{itemize}}{\end{itemize}}
