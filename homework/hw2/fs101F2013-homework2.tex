\input{hwpre.tex}

\usepackage[compact]{titlesec}

\begin{document}
\MYTITLE{Short Writing Assignment: Managing Your Time}
\MYHEADERS{Short Writing Assignment}{Managing Your Time}

\vspace*{-.2in}
\begin{center}
	{\bf Due Date: Wednesday, September 23, 2013}
\end{center}

In order to ensure that you have a smooth start to your academic career at Allegheny College, it is useful, among other
things, to organize your weekly schedule. In this assignment, you must create your own schedule using Google Calendar.
After visiting the Web site \url{http://sites.allegheny.edu/my/}, click on the box with the label ``Calendar'' and the
icon of a calendar. Including as many details as is possible, start to add in the different events that you will attend
during the week. For example, your schedule should include all of your class and laboratory sessions, the times during
which you will study, the meeting times for the clubs and organizations to which you belong, when you will meet with
professors during office hours, and any other relevant activities.

Next, you should print a one page summary of your schedule that you can turn in as a part of this assignment. You should
also explore all of the different options that are associated with configuring your Google Calendar (e.g., security
permissions, color coding, and multiple calendars). After completing and printing your schedule, please reflect on the
process that you followed to produce your calendar. As the final part of your assignment, you should write a one to two
page tutorial that clearly explains how an Allegheny College student could create their own schedule in Google Calendar.
The document must precisely describe the steps that could easily be followed by a student who is new to Allegheny
College. You may consider including an appendix of screenshots as one way to make your tutorial more concrete and easier
to understand.

All students are encouraged to bring a draft of their materials to the Learning Commons and/or the instructor's office
hours in order to receive feedback on their writing assignment.

\end{document}
